\chapter{Conclusions}
\thispagestyle{empty}% no page number in chapter title page

The goal of this work was to create a parallel algorithm which should simplify planar surfaces and at the same time preserve high level of details in areas with complex geometry. The results show that the goal was achieved. Complex shapes are almost entirely preserved, whereas, planar surfaces are described using few faces. Appendix A shows the results where the best approximation is generated using quadric error metric with all vertex's attributes. However, at the same time, this metric is the slowest one because of the number of parameters to jointly optimize for every edge.

Summarizing, the algorithm is able to generate high quality progressive meshes in reasonable time, which was one of the most important aspects of this work. The results are promising and in some cases are better than commercially available products. The time of processing is the biggest flaw of the approach. This problem can be reduced by increasing aggressiveness, however, the quality and our assumptions will suffer from it. Despite the fact that parallelization gives in the first iteration almost 4 times speedup, all approximations have to be generated beforehand for the streaming purposes.
